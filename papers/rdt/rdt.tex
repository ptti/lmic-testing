\documentclass{amsart}
\usepackage{marginnote}
\usepackage{tikz}
\usepackage{epi}
\usepackage{lshtm}

\colorlet{S}{LSHTMBlue}
\colorlet{E}{LSHTMPurple}
\colorlet{Ip}{LSHTMYellow}
\colorlet{I}{LSHTMOrange}
\colorlet{R}{LSHTMGreen}
\colorlet{D}{black}

\title{Use of rapid diagnostic tests}

\begin{document}

\section{Methods}

\begin{figure}
  \begin{tikzpicture}
    \node[agent,fill=S!20] (S) at (0,0) { $P_{S}$ };
    \node[agent,fill=E!20] (E) at (2,0) { $P_{E}$ };
    \node[agent,fill=Ip!20] (Ip) at (4,0) { $P_{I,c(P)}$ };
    \node[agent,fill=I!30] (Ia) at (6,1) { $P_{I,c(A)}$ };
    \node[agent,fill=I!40] (Is) at (6,-1) { $P_{I,c(S)}$ };
    \node[agent,fill=I!50] (Ic) at (8,-2) { $P_{I,c(S),o(C)}$ };
    \node[agent,fill=R!20] (R) at (10,1) { $P_{R}$ };
    \node[agent,fill=D!20] (D) at (10,-1) { $P_{o(D)}$ };

    \path[draw,->] (S) -- (E);
    \path[draw,->] (E) -- (Ip);
    \path[draw,->] (Ip) -- (Ia);
    \path[draw,->] (Ip) -- (Is);
    \path[draw,->] (Ia) -- (R);
    \path[draw,->] (Is) -- (Ic);
    \path[draw,->] (Is) -- (R);
    \path[draw,->] (Ic) -- (R);
    \path[draw,->] (Ic) -- (D);
  \end{tikzpicture}
\end{figure}


\subsection{Natural history}


\subsection{Calibration}
We begin by establishing a baseline infectiousness, $\beta$, which we will
subsequently fix. We do this by simulating three weeks of the exponential growth
phase of an unmitigated epidemic similar to what occurred in the UK and
elsewhere in the early stages of the COVID-19 pandemic. We know that in the
before times the average contact rate between individuals in the UK was
approximately 12 per person per day. We therefore fix $c = 12$. We believe that
the basic reproduction number at that time in that epidemic was approximately 3.
We simulate a sufficiently large population of $10^5$ individuals of whom $10^2$
are initially infectious, this size being chosen so that the epidemic does not
leave the exponential growth regime in the first 21 days. We therefore fit
$\beta$ under these conditions. \marginnote{\textbf{Assumption:} We can
  establish baseline infectiousness for   this model with a contact rate of 12
  and a basic reproduction number of 3.} We obtain a mean value for $\beta$ of
0.037 with a standard deviation of less than $10^{-5}$.

Next, we check this by producing a model with a basic reproduction number of one
half the above, $R(0) = 1.5$. We fix $\beta$ to the value that we have found and
allow the contact rate to vary. As we would expect from the analytical
calculation for the mean-field approximation of this model, we find a value for
$c$ of 5.99 with a standard deviation of $2\times 10^{-4}$ or, essentially 6.



\end{document}