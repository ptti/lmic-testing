\documentclass{amsart}
\usepackage{marginnote}

\title{Use of rapid diagnostic tests}

\begin{document}

\section{Methods}

We begin by establishing a baseline infectiousness, $\beta$, which we will
subsequently fix. We do this by simulating three weeks of the exponential growth
phase of an unmitigated epidemic similar to what occurred in the UK and
elsewhere in the early stages of the COVID-19 pandemic. We know that in the
before times the average contact rate between individuals in the UK was
approximately 12 per person per day. We therefore fix $c = 12$. We believe that
the basic reproduction number at that time in that epidemic was approximately 3.
We therefore fit $\beta$ under these conditions.
\marginnote{\textbf{Assumption:} We can establish baseline infectiousness for
  this model with a contact rate of 12 and a basic reproduction number of 3.}
\end{document}